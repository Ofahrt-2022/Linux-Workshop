\documentclass[
    ngerman,
    accentcolor=3b,
    dark_mode,
    fontsize= 12pt,
    a4paper,
    aspectratio=169,
    colorback=true,
    fancy_row_colors,
    leqno,
    fleqn,
    boxarc=3pt,
    fleqn,
    % shell_escape = false, % Kompatibilität mit sharelatex
]{algoslides}

%%------------%%
%%--Packages--%%
%%------------%%

% \usepackage{audutils}
% \usepackage{fopbot}
\usepackage{booktabs}

% Import all Packages from Main Preamble with relative Path (buggy, list packages instead)
% \subimport*{../../}{preamble}

%%--------------------------%%
%%--Imports from Main File--%%
%%--------------------------%%

% Get Labels from Main Document using the xr-hyper Package
\externaldocument[ext:]{../main}
% Set Graphics Path, so pictures load correctly
\graphicspath{{../pictures}}

\begin{document}
    \section{Paketverwaltung}\label{package-management}\label{Paketverwaltung}
    \subsection{Die Paketmanager großer Distros}
    \begin{frame}
        \slidehead{}
        \begin{itemize}
            \item Redhat based: {\ttfamily yum}
            \item Debian based: {\ttfamily apt}
            \item Arch based: {\ttfamily pacman}
        \end{itemize}
    \end{frame}
    \subsection{Vorteile der Nutzung eines Paketmanager}
    \begin{frame}
        \slidehead{}
        \begin{itemize}
            \item Software kann zentral installiert, konfiguriert und entfernt werden
            \item Software kommt aus vertrauter Quelle
            \item Abhängigkeiten werden automatisch mit installiert
            \item Nicht jedes Software muss ihren eigenen Updater mitbringen
        \end{itemize}
    \end{frame}
    \subsection{Die wichtigsten Befehle: Pakete updaten}
    \begin{frame}[fragile]
        \slidehead{}
        \begin{codeBlock}[fontsize=\tiny]{minted language=text, title=\codeBlockTitle{Pakete updaten mit {\ttfamily yum}}}
            [root@oshgnacknak ~]# yum update
            Loaded plugins: fastestmirror
            Loading mirror speeds from cached hostfile
             * base: mirror.imt-systems.com
             * epel: mirror.imt-systems.com
             * extras: ftp.plusline.net
             * updates: de.mirrors.clouvider.net
            No packages marked for update
        \end{codeBlock}
    \end{frame}
    \begin{frame}[fragile]
        \slidehead{}
        \begin{codeBlock}[fontsize=\tiny, escapeinside=§§]{minted language=text, title=\codeBlockTitle{Pakete updaten mit {\ttfamily apt}}}
            pi@naspi:~ $ sudo apt update
            Hit:1 http://raspbian.raspberrypi.org/raspbian buster InRelease
            Hit:2 http://archive.raspberrypi.org/debian buster InRelease
            Get:3 https://download.docker.com/linux/raspbian buster InRelease [33.6 kB]
            Hit:4 https://deb.nodesource.com/node_14.x buster InRelease
            Fetched 33.6 kB in 1s (27.9 kB/s)
            Reading package lists... Done
            Building dependency tree
            Reading state information... Done
            All packages are up to date.
            pi@naspi:~ $ sudo apt upgrade
            Reading package lists... Done
            Building dependency tree
            Reading state information... Done
            Calculating upgrade... Done
            Use 'sudo apt autoremove' to remove them.
            0 upgraded, 0 newly installed, 0 to remove and 0 not upgraded.
        \end{codeBlock}
    \end{frame}
    \begin{frame}[fragile]
        \slidehead{}
        \begin{codeBlock}[fontsize=\tiny]{minted language=text, title=\codeBlockTitle{Pakete updaten mit {\ttfamily pacman}}}
            [osh@oshlaptop ~]$ sudo pacman -Suy
            :: Paketdatenbanken werden synchronisiert …
             core ist aktuell
             extra                                           1700,9 KiB  1001 KiB/s 00:02 [######################] 100%
             community                                          6,6 MiB  1375 KiB/s 00:05 [######################] 100%
             multilib ist aktuell
            :: Vollständige Systemaktualisierung wird gestartet …
             Es gibt nichts zu tun
        \end{codeBlock}
    \end{frame}
    \subsection{Die wichtigsten Befehle: Pakete suchen}
    \begin{frame}[fragile]
        \slidehead{}
        \begin{codeBlock}[fontsize=\tiny]{minted language=text, title=\codeBlockTitle{Pakete suchen mit {\ttfamily yum}}}
            [root@oshgnacknak ~]# yum search gcc
            Loaded plugins: fastestmirror
            Loading mirror speeds from cached hostfile
             * base: mirror.imt-systems.com
             * epel: mirror.imt-systems.com
             * extras: ftp.plusline.net
             * updates: de.mirrors.clouvider.net
            ================================= N/S matched: gcc ==================================
            avr-gcc.x86_64 : Cross Compiling GNU GCC targeted at avr
            avr-gcc-c++.x86_64 : Cross Compiling GNU GCC targeted at avr
            csgcca.x86_64 : A compiler wrapper that runs 'gcc -fanalyzer' in background
            gcc-c++.x86_64 : C++ support for GCC
            gcc-gnat.x86_64 : Ada 95 support for GCC
            gcc-objc.x86_64 : Objective-C support for GCC
            gcc-objc++.x86_64 : Objective-C++ support for GCC
            gcc-plugin-devel.x86_64 : Support for compiling GCC plugins
        \end{codeBlock}
    \end{frame}
    \begin{frame}[fragile]
        \slidehead{}
        \begin{codeBlock}[fontsize=\tiny]{minted language=text, title=\codeBlockTitle{Pakete suchen mit {\ttfamily apt}}}
            pi@naspi:~ $ apt search gcc
            Sorting... Done
            Full Text Search... Done
            arduino-builder/oldstable 1.3.25-1 armhf
              Command line tool for compiling Arduino sketches

            autoconf2.59/oldstable 2.59+dfsg-1 all
              automatic configure script builder (obsolete version)

            autoconf2.64/oldstable 2.64+dfsg-1 all
              automatic configure script builder (obsolete version)

            autofdo/oldstable 0.18-2 armhf
              AutoFDO Profile Toolchain

            binutils-mingw-w64/oldstable 2.31.1-11+8.3 all
              Cross-binutils for Win32 and Win64 using MinGW-w64
        \end{codeBlock}
    \end{frame}
    \begin{frame}[fragile]
        \slidehead{}
        \begin{codeBlock}[fontsize=\tiny]{minted language=text, title=\codeBlockTitle{Pakete suchen mit {\ttfamily pacman}}}
            $ pacman -Ss gcc
            core/gcc 11.2.0-4 (base-devel) [Installiert]
                The GNU Compiler Collection - C and C++ frontends
            core/gcc-ada 11.2.0-4
                Ada front-end for GCC (GNAT)
            core/gcc-d 11.2.0-4
                D frontend for GCC
            core/gcc-fortran 11.2.0-4
                Fortran front-end for GCC
            core/gcc-go 11.2.0-4
                Go front-end for GCC
            core/gcc-libs 11.2.0-4 [Installiert]
                Runtime libraries shipped by GCC
            core/gcc-objc 11.2.0-4
                Objective-C front-end for GCC
        \end{codeBlock}
    \end{frame}
    \subsection{Die wichtigsten Befehle: Pakete installieren}
    \begin{frame}[fragile]
        \slidehead{}
        \begin{codeBlock}[fontsize=\tiny]{minted language=text, title=\codeBlockTitle{Pakete installieren mit {\ttfamily yum}}}
            [root@oshgnacknak ~]# yum install gcc
            Loaded plugins: fastestmirror
            Loading mirror speeds from cached hostfile
             * base: mirror.imt-systems.com
             * epel: mirror.imt-systems.com
             * extras: ftp.plusline.net
             * updates: de.mirrors.clouvider.net
            Package gcc-4.8.5-44.el7.x86_64 already installed and latest version
            Nothing to do
        \end{codeBlock}
    \end{frame}
    \begin{frame}[fragile]
        \slidehead{}
        \begin{codeBlock}[fontsize=\tiny]{minted language=text, title=\codeBlockTitle{Pakete installieren mit {\ttfamily apt}}}
            pi@naspi:~ $ sudo apt install gcc
            Reading package lists... Done
            Building dependency tree
            Reading state information... Done
            gcc is already the newest version (4:8.3.0-1+rpi2).
            Use 'sudo apt autoremove' to remove them.
            0 upgraded, 0 newly installed, 0 to remove and 0 not upgraded.
        \end{codeBlock}
    \end{frame}
    \begin{frame}[fragile]
        \slidehead{}
        \begin{codeBlock}[fontsize=\tiny]{minted language=text, title=\codeBlockTitle{Pakete installieren mit {\ttfamily pacman}}}
            $ sudo pacman -S gcc --needed
            Warnung: gcc-11.2.0-4 ist aktuell -- Überspringe
            Es gibt nichts zu tun
        \end{codeBlock}
    \end{frame}
    \subsection{Die wichtigsten Befehle: Pakete entfernen}
    \begin{frame}[fragile]
        \slidehead{}
        \begin{codeBlock}[fontsize=\tiny]{minted language=text, title=\codeBlockTitle{Pakete entfernen mit {\ttfamily yum}}}
            [root@oshgnacknak ~]# yum remove gcc
            Loaded plugins: fastestmirror
            No Match for argument: gcc
            No Packages marked for removal
        \end{codeBlock}
    \end{frame}
    \begin{frame}[fragile]
        \slidehead{}
        \begin{codeBlock}[fontsize=\tiny]{minted language=text, title=\codeBlockTitle{Pakete entfernen mit {\ttfamily apt}}}
            pi@naspi:~ $ sudo apt remove gcc
            Reading package lists... Done
            Building dependency tree
            Reading state information... Done
            The following packages were automatically installed and are no longer required:
              g++-8 libstdc++-8-dev
            Use 'sudo apt autoremove' to remove them.
            The following packages will be REMOVED:
              build-essential g++ gcc
            0 upgraded, 0 newly installed, 3 to remove and 0 not upgraded.
            After this operation, 81.9 kB disk space will be freed.
            Do you want to continue? [Y/n]
            (Reading database ... 54461 files and directories currently installed.)
            Removing build-essential (12.6) ...
            Removing g++ (4:8.3.0-1+rpi2) ...
            Removing gcc (4:8.3.0-1+rpi2) ...
            Processing triggers for man-db (2.8.5-2) ...
        \end{codeBlock}
    \end{frame}
    \begin{frame}[fragile]
        \slidehead{}
        \begin{codeBlock}[fontsize=\tiny]{minted language=text, title=\codeBlockTitle{Pakete entfernen mit {\ttfamily pacman}}}
            [osh@oshlaptop ~]$ sudo pacman -Rns gcc
            Fehler: Ziel nicht gefunden: gcc
        \end{codeBlock}
    \end{frame}
\end{document}
