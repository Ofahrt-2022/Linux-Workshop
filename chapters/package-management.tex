\documentclass[
    ngerman,
    accentcolor=3b,
    dark_mode,
    fontsize= 12pt,
    a4paper,
    aspectratio=169,
    colorback=true,
    fancy_row_colors,
    leqno,
    fleqn,
    boxarc=3pt,
    fleqn,
    % shell_escape = false, % Kompatibilität mit sharelatex
]{algoslides}

%%------------%%
%%--Packages--%%
%%------------%%

% \usepackage{audutils}
% \usepackage{fopbot}
\usepackage{booktabs}

% Import all Packages from Main Preamble with relative Path (buggy, list packages instead)
% \subimport*{../../}{preamble}

%%--------------------------%%
%%--Imports from Main File--%%
%%--------------------------%%

% Get Labels from Main Document using the xr-hyper Package
\externaldocument[ext:]{../main}
% Set Graphics Path, so pictures load correctly
\graphicspath{{../pictures}}

\begin{document}
    \section{Paketverwaltung}\label{package-management}\label{Paketverwaltung}
    \subsection{Die Paketmanager großer Distros}
    \begin{frame}
        \slidehead{}
        \begin{itemize}
            \item Redhat based: {\ttfamily yum}
            \item Debian based: {\ttfamily apt}
            \item Arch based: {\ttfamily pacman}
        \end{itemize}
    \end{frame}
    \subsection{Vorteile der Nutzung eines Paketmanager}
    \begin{frame}
        \slidehead{}
        \begin{itemize}
            \item Software kann zentral installiert, konfiguriert und entfernt werden
            \item Software kommt aus vertrauter Quelle
            \item Abhängigkeiten werden automatisch mit installiert
            \item Nicht jedes Software muss ihren eigenen Updater mitbringen
        \end{itemize}
    \end{frame}
    \subsection{Die wichtigsten Befehle: Pakete suchen}
    \begin{frame}[fragile]
        \slidehead{}
        \begin{columns}
            \begin{column}[c]{\textwidth/3}
                \begin{codeBlock}[]{}
                    a + a = 2*a
                \end{codeBlock}
            \end{column}
            \begin{column}[c]{\textwidth/3}
                b
            \end{column}
            \begin{column}[c]{\textwidth/3}
                c
            \end{column}
        \end{columns}
    \end{frame}
\end{document}
