\documentclass[
    ngerman,
    accentcolor=3b,
    dark_mode,
    fontsize= 12pt,
    a4paper,
    aspectratio=169,
    colorback=true,
    fancy_row_colors,
    leqno,
    fleqn,
    boxarc=3pt,
    fleqn,
    main,
    % shell_escape = false, % Kompatibilität mit sharelatex
    ]{algoslides}

%%------------%%
%%--Packages--%%
%%------------%%

% \usepackage{audutils}
% \usepackage{fopbot}
\usepackage{booktabs}


%%----------------------------%%
%%--Stilistische Anpassungen--%%
%%----------------------------%%


\renewcommand\tabularxcolumn[1]{m{#1}}% for vertical centering text in X column
% Remove unwanted space from tables
\aboverulesep = 0mm \belowrulesep = 0mm
\renewcommand{\arraystretch}{1.4}

%%---------------------------%%
%%--Dokumenteneinstellungen--%%
%%---------------------------%%

% Workshop-Individuelle Einstellungen
\def\workshoptitle{<Workshoptitel>}
\def\shortworkshoptitle{<Kurztitel>}
\subtitle{<Subtitel>}
\def\gruesswoerte{<Begrüßende Worte>}
\author{Ruben Deisenroth}
\titlegraphic*{\includegraphics{example-image}}


% Ofahrt-Einstellungen
\graphicspath{{pictures}}
\title[\shortworkshoptitle{}]{\workshoptitle{} -- Ofahrt 2022}
\department{TU Darmstadt | Fachbereich Informatik | Ofahrt 2022 | \insertshorttitle{}}
\date{\today}
\logo*{\includegraphics{example-image-16x9}} % TODO: Ofahrt-Logo erstellen

%%-------------------------%%
%%--Beginn des Dokumentes--%%
%%-------------------------%%

\begin{document}

    %%-----------%%
    %%--Titelei--%%
    %%-----------%%

    \maketitle{}

    \begin{frame}[c]
        % Die Begrüßenden Worte können individuell pro Workshop festgelegt werden. Am Besten nur was kurzes, sowie "Gude", "Hi", "Herzlich Willkommen!", ...
        \centering\huge\textbf{\gruesswoerte{}}
    \end{frame}

    %%---------------------------%%
    %%--Beginn der Präsentation--%%
    %%---------------------------%%
    
    \begin{frame}
        \frametitle{Das steht heute auf dem Plan}
        \tableofcontents[subsubsectionstyle=hide]
    \end{frame}


    %%-----------%%
    %%--Kapitel--%%
    %%-----------%%

    \documentclass[
    ngerman,
    accentcolor=3b,
    dark_mode,
    fontsize= 12pt,
    a4paper,
    aspectratio=169,
    colorback=true,
    fancy_row_colors,
    leqno,
    fleqn,
    boxarc=3pt,
    fleqn,
    % shell_escape = false, % Kompatibilität mit sharelatex
    ]{algoslides}

%%------------%%
%%--Packages--%%
%%------------%%

% \usepackage{audutils}
% \usepackage{fopbot}
\usepackage{booktabs}

% Import all Packages from Main Preamble with relative Path (buggy, list packages instead)
% \subimport*{../../}{preamble}

%%--------------------------%%
%%--Imports from Main File--%%
%%--------------------------%%

% Get Labels from Main Document using the xr-hyper Package
\externaldocument[ext:]{../main}
% Set Graphics Path, so pictures load correctly
\graphicspath{{../pictures}}

\begin{document}
\section{Einführung}\label{1}\label{Einfuehrung}
\begin{frame}
    \slidehead{}
\end{frame}
\end{document}
\end{document}
